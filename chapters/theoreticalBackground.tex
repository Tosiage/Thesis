\chapter{\iflanguage{english}{Theoretical Background}{Grundlagen}}
\label{cha:theoreticalBackground}

Description of theoretical concepts which are required to understand the rest of the thesis.

Describe the concepts in enough detail so that other computer scientists can understand your methods in the next chapter. Go into enough detail so that the concept can be easily understood, but don't write a reference book! Instead, give proper sources and citations so the reader can look up further details if they want to.



\section{Mixed Reality}

%Bild einfügen MR Spektrum

Der Begriff \emph{Mixed Reality} (MR) wurde zuerst von Milgram und Kishino \cite{milgram_taxonomy_1994} beschrieben. Dabei handelt es sich um ein Spektrum, welches sich zwischen der realen und der virtuellen Welt aufspannt. Unter der virtuellen Welt wird hierbei die \emph{Virtuelle Realität} (VR) verstanden, welche sich dadurch kennzeichnet, dass der Nutzer vollständig in eine digitalisierte Umgebung eintaucht, ohne seine reale Umwelt direkt wahrnehmen zu können. Auf dem MR Spektrum sind reale und virtuelle Welt unterschiedich stark miteinander verschmolzen. Bei der \emph{Augmented Reality} (AR) werden der realen Welt virtuelle Objekte, wie bspw. der Avatar einer sich nicht im Raum befindlichen Person, hinzugefügt \cite{bray_what_nodate}. Bei der \emph{Augmented Virtuality} (AV) werden ausgehend von der virtuellen Welt Eigenschaften des realen Raumes übernommen, wie etwa Hindernisse \cite{bray_what_nodate}. 

Heutzutage wird die MR nicht nur durch die Art des verwendeten Bildschirms beschrieben, sondern zusätzlich durch die Fähigkeit des Geräts, sich im Raum lokalisieren zu können und bspw. Geräusche räumlich korrekt widergeben zu können \cite{bray_what_nodate}. 

%Platzhalter
%TODO neue Graphik

\begin{figure}[ht]
	\centering
		\includegraphics[width=0.75\textwidth]{images/IEICEfig1.png}
	\caption[Virtuality Continuum]{Mixed Reality Spektrum nach Milgram und Kishino \cite{milgram_taxonomy_1994}}
	\label{fig:4.1}
\end{figure}
 

\section{Head Mounted Displays}
-Video Durchsicht Brille
-Optische Durchsicht

\subsection{Microsoft Hololens}
-specs
-Wie findet sich die hololens im raum zurecht (inside out tracking)
-koordinatensysteme
-unity
-MRToolkit

\section{AR Tracking}
-was sind marker
-verschiedene marker arten
-optical marker

\subsection{Konkrete Marker Arten}

\subsection{Vuforia}
-model targets
-image targets

\section{Phantom}

\section{Registrierung}