\chapter{\iflanguage{english}{Theoretical Background}{Grundlagen}}
\label{cha:theoreticalBackground}

%TODO
Description of theoretical concepts which are required to understand the rest of the thesis.

Describe the concepts in enough detail so that other computer scientists can understand your methods in the next chapter. Go into enough detail so that the concept can be easily understood, but don't write a reference book! Instead, give proper sources and citations so the reader can look up further details if they want to.



\section{Mixed Reality}

%Bild einfügen MR Spektrum

Der Begriff \emph{Mixed Reality} (MR) wurde zuerst von Milgram und Kishino \cite{milgram_taxonomy_1994} beschrieben. Dabei handelt es sich um ein Spektrum, welches sich zwischen der realen und der virtuellen Welt aufspannt. Unter der virtuellen Welt wird hierbei die \emph{Virtuelle Realität} (VR) verstanden, welche sich dadurch kennzeichnet, dass der Nutzer vollständig in eine digitalisierte Umgebung eintaucht, ohne seine reale Umwelt direkt wahrnehmen zu können. Auf dem MR Spektrum sind reale und virtuelle Welt unterschiedich stark miteinander verschmolzen. Bei der \emph{Augmented Reality} (AR) werden der realen Welt virtuelle Objekte, wie bspw. der Avatar einer sich nicht im Raum befindlichen Person, hinzugefügt \cite{bray_what_nodate}. Bei der \emph{Augmented Virtuality} (AV) werden ausgehend von der virtuellen Welt Eigenschaften des realen Raumes übernommen, wie etwa Hindernisse \cite{bray_what_nodate}. 

Heutzutage wird die MR nicht nur durch die Art des verwendeten Bildschirms beschrieben, sondern zusätzlich durch die Fähigkeit des Geräts, sich im Raum lokalisieren zu können und bspw. Geräusche räumlich korrekt widergeben zu können \cite{bray_what_nodate}. 

%Platzhalter
%TODO neue Graphik
%TODO Verweis in Text auf Abbildung

\begin{figure}[ht]
	\centering
		\includegraphics[width=0.75\textwidth]{images/IEICEfig1.png}
	\caption[Virtuality Continuum]{Mixed Reality Spektrum nach Milgram und Kishino \cite{milgram_taxonomy_1994}}
	\label{fig:4.1}
\end{figure}
 

\section{Head Mounted Displays}

Um virtuelle Objekte in der realen Welt anzeigen zu können, werden Displays benötigt. Neben Displays, welche in der Hand gehalten werden und stationär befestigten Displays, gibt es \emph{Head Mounted Displays} (HMDs). Sie werden ähnlich einer Brille auf dem Kopf getragen, das Display befindet sich vor den Augen des Nutzers. 
Im Bereich AR wird zwischen \emph{video see-through HMD} (VST-HMD) und \emph{optical see-through HMD} (OST-HMD) unterschieden.

Bei VST-HMDs wird ein Video der Umwelt aufgenommen, in welches die virtuellen Objekte mittels Methoden der Computer Vision pixelgenau eingefügt werden \cite{billinghurst_survey_2015}. Es tritt keine Zeitverzögerung zwischen Bild der realen Welt und dem darauf angezeigtem virtuellem Objekt auf. Mittels Tiefenkameras kann eine korrekte räumliche Überlagerung von virtuellen Objekten durch reale Objekte garantiert werden. Ein Nachteil von VST-HMDs ist die Tatsache, dass der Träger des HMDs die Welt nur über das Video wahrnimmt und es so zu Problemen aufgrund von begrenzter Auflösung und zeitlicher Verzögerung zwischen realer Welt und Video kommen kann \cite{billinghurst_survey_2015}.  

OST-HMDs hingegen sind Displays, welche durchsichtig sind und so den Blick auf die reale Umgebung ermöglichen. Dadurch gibt es keine Zeitverzögerung zwischen realer Welt und Szene, die der Nutzer sieht, mehr. Allerdings kann es nun zur zeitlichen Verzögerung im Rendering des virtuellen Objektes kommen. Ein weiteres großes Problem ist das Registrieren der virtuellen Objekte mit der Umwelt, da bei OST-HMDs zum positionieren dieser Objekte keine Methoden der Computer Vision verwendet werden können \cite{billinghurst_survey_2015}. Die in dieser Arbeit verwendete Microsoft Hololens lässt sich zu den OST-HMDs zählen. 








\section{AR Tracking}
-was sind marker
-verschiedene marker arten
-optical marker

\subsection{Konkrete Marker Arten}



\section{Phantom}

\section{Registrierung}