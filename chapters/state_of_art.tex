\chapter{\iflanguage{english}{State of the Art}{Stand der Forschung}}
\label{cha:stateOfTheArt}

Im Folgenden werden verwandte Arbeiten genannt, wobei besonderer Fokus auf Anwendungen mit der Microsoft Hololens gelegt wird. Weiterhin wird auf verschiedene Möglichkeiten der Registrierung, u.a. durch Marker Tracking eingegangen. 


\section{Augmented Reality in der Chirurgie}

Augmented Reality kann im medizinischen Bereich in vielerlei Hinsicht eingesetzt werden \cite{shuhaiber_augmented_2004}. Großes Potential haben Anwendungen zur präoperativen Planung \cite{ha_augmented_2016}, intraoperativer Visualisierung und zum Training von Chirurgen \cite{azuma_survey_1997}. Daten können entweder direkt auf dem Videostream eingeblendet und auf einem Bildschirm angezeigt werden oder mit Hilfe eines \emph{Head Mounted Displays} (HMD), also einem Bildschirm, welcher wie eine Brille auf dem Kopf getragen wird, visualisiert werden. Anwendungen mit HMDs bieten zusätzlich den Vorteil, dass der Chirurg seine Hände frei nutzen kann im Gegensatz zu einer Anzeige auf bspw. einem Tablet.  Eine Herausforderung hierbei ist, dass zur Visualisierung die Daten korrekt mit dem Patienten überlagert (\emph{registriert}) werden müssen. Es muss also erreicht werden, dass die wahrgenommene Position in der Szene mit dem computergenerierten Objekt übereinstimmt \cite{birkfellner_head-mounted_2002}. Dazu muss die Kamera kalibriert werden. 

Azuma \cite{azuma_survey_1997} hat bereits 1997 Anwendungsfälle für AR im medizinischen Bereich genannt, unter anderem ein Projekt, bei welchem eine Biopsie des Brustgewebes durch AR unterstützt wird. Die Position des Tumors in einem Phantom wird visualisiert und die Einstichstelle sowie der Pfad der Nadel im Gewebe bis zum Tumor geführt.

Birkfellner et al. \cite{birkfellner_head-mounted_2002} haben ein eigenes HMD entwickelt, indem sie ein Operationsmikroskop um AR-Fähigkeit erweitert haben.
Bei \emph{INPRES} (intraoperative presentation of surgical planning and simulation results)\cite{salb_inpres_2003} wird ein System zur Unterstützung von Chirurgen bei kraniofazialen Operationen entwickelt. Als HMD wird die Optische-Durchsicht-Brille Sony Glasstron verwendet. Das HMD sowie Patient und Instrumente werden mit Hilfe eines Infrarot-Trackingsystems im Raum gefunden. So wird eine Registrierung von virtuellen Objekten mit der realen Welt erreicht. Mit Hilfe von zusätzlich angebrachten Kameras wird eine Erkennung von Verdeckungen in der Objektüberlagerung implementiert. 

\section{Marker Tracking in der Chirurgie}

Ziel dieser Arbeit ist, ein Modell eines Phantoms in der virtuellen Welt zum Phantom in der realen Welt zu registrieren. Dazu muss die Lage des Phantoms in der realen Welt bekannt sein. Die Bestimmung der Lage des Phantoms kann mit Hilfe von \emph{Fiducials} (Marker) geschehen. Marker sind Objekte, welche am Phantom angebracht werden und in der realen Welt detektiert werden können. Marker können in der virtuellen Welt kalibriert werden, sodass die Transformation eines Markers zum Phantom in der realen Welt bekannt ist. So kann das Tracking des Phantoms über das Tracking von Markern geschehen, wenn es nicht möglich ist, das Phantom selbst in der realen Welt zu detektieren. Es existieren aktive (bspw. elektromagnetische) Marker, welche ein Signal aussenden, sowie passive Marker und verschiedene Methoden, diese Marker zu detektieren. In dieser Arbeit werden passive Marker genutzt, welche in Kamerabildern gefunden werden.

Ha und Hong \cite{ha_augmented_2016} haben ein System zur Unterstützung einer Knochentumorresektion entwickelt. Um den Patienten und die Instrumente zu tracken, wird ein Tablet Computer mit Kamera genutzt, auf dem auch gleichzeitig die Visualisierung zu sehen ist. Als Marker werden Würfel genutzt, an welchen auf jeder Seite ein planares Bild angebracht ist. Diese Marker werden am Patienten und an den Instrumenten befestigt, ein externes Tracking System ist nicht notwendig. 

Kenngott et al. \cite{kenngott_mobile_2018} haben ein mobiles System entwickelt, welches eine möglichst patientennahe Anwendung von Augmented Reality ermöglicht. So können Aufnahmen des Patienten schnellstmöglich visualisiert werden. Dies ist vor allem in Umfeldern, wo es auf Schnelligkeit ankommt (bspw. in der Notaufnahme) ein großer Vorteil gegenüber des üblichen Vorgehens, dem Anzeigen der Aufnahmen an einem dafür vorgesehenen Arbeitsplatz. Um den Bezugsrahmen für die AR Anwendung zu definieren, werden 15 strahlenundurchlässige Marker genutzt, welche frei auf dem Patienten positioniert werden können und während der CT Aufnahme am Patienten angebracht sind. Für den Registrierungsprozess werden 6 dieser Marker genutzt und farblich markiert. Zum Tracking der Marker und Anzeigen des Hologramms wird ein handelsübliches Apple iPad genutzt.

Frantz et al. \cite{frantz_augmenting_2018} nutzen Cylinder Targets für eine Registrierung eines Schädels mit einem virtuellen Objekt. Dabei handelt es sich um ein Bild, welches außen an einem Zylinder angebracht wird und getrackt werden kann auch wenn der Nutzer sich um den Zylinder herum bewegt.


\section{Hololens in der Chirurgie}

Die Microsoft Hololens ist ein HMD für Augmented Reality und laut Pratt et al. \cite{pratt_through_2018} zurzeit eines der besten HMDs auf dem Markt. Es ist ein optical see-through display, ermöglicht also dem Chirurgen gleichzeitig die Sicht auf die reale Umgebung ohne eine Zeitverzögerung. Die Hololens ist nicht kabelgebunden und lässt sich über Handgesten und Sprachbefehle steuern, was die Sterilität im Operationssaal nicht gefährdet \cite{pratt_through_2018}.

\begin{figure}[h!]
	\centering
		\includegraphics[width=0.75\textwidth]{images/PrattAR.png}
	\caption[AR-gestützte Extremitätenrekonstruktion]{Hololensgestützte Operation; (\textbf{a}) Blick auf OP über zweite Hololens; (\textbf{b}) Chirurg nutzt Doppler-Sonografie zur Überprüfung; (\textbf{c}) Registrierung von Hologramm und Patient, Hervorhebung wichtiger Positionen durch Pfeile \cite{pratt_through_2018}}
	\label{fig:2.1}
\end{figure}

Von Pratt et al. \cite{pratt_through_2018} wird eine Hololens App zur Unterstützung der Chirurgen bei einer Operation zur Rekonstruktion von Extremitäten entwickelt (siehe Abb. \ref{fig:2.1}). Es wird festgestellt, dass die Anwendung verlässlicher und weniger zeitaufwändig ist, als die zur Zeit für solche Eingriffe genutzte Doppler-Sonografie. Das Hologramm wird beim Start der App ausgehend von der Pose der Hololens an einer festgelegten Stelle im Raum gerendert. Die Registrierung wird vom Chirurgen manuell durch das nutzen der Air-Tap-and-Hold Geste durchgeführt. Über Sprachbefehle oder entsprechende Buttons in einer Toolbar kann zwischen Translation und Rotation Modus gewechselt werden. 


Liebmann et al. \cite{liebmann_pedicle_2019} haben eine Navigationsanwendung für die Platzierung von Pedikelschrauben bei Operationen an der Wirbelsäule entwickelt. Ziel der Anwendung ist, intraoperative Bildgebung wie CT Aufnahmen zu vermeiden, da der Patient dabei einer hohen Strahlenbelastung ausgesetzt wird. Mit 2 der 4 Frontkameras der Hololens wird Marker Tracking zur Bestimmung der Kamerapose umgesetzt. Als Marker werden sterile kommerziell verfügbare Marker verwendet. Um zu Registrieren, wird intraoperativ die Oberfläche der Wirbelsäule mit Hilfe eines Pointers (siehe Abb.\ref{fig:2.1}(a)), der getrackt werden kann, abgetastet. Zusätzlich wird der Klicker der Hololens genutzt, welcher einen Button besitzt und Anstelle von Handgesten zur Interaktion genutzt werden kann. Während der Klicker gedrückt wird, wird die Position vom Pointer gespeichert. Die so erhaltene intraoperative Punktwolke wird anschließend vollautomatisch mit der präoperativen Punktwolke registriert (siehe Abb. \ref{fig:2.1}(d)). Abschließend wird das Hologramm angezeigt und bei korrekter Überlagerung vom Chirurgen bestätigt.  Durch doppelklicken des Hololens Klickers wird die Navigation gestartet. Während der Navigation wird das Hologramm ausgeblendet und nur die Bohrlöcher visualisiert, zusätzlich wird dem Chirurgen die aktuelle und gewünschte Neigung eines Navigationswerkzeuges angezeigt.

Rae et al. \cite{rae_neurosurgical_2018} nutzen die Hololens, um die Platzierung von Bohrlöchern auf dem Schädel anzuzeigen. Mit dem Tool 3D Slicer wird ein 3D Modell erstellt und Marker sowohl an den gewünschten Bohrpunkten als auch an Registrierungspunkten platziert. Die Registrierung des 3D Modells via Hololens mit dem Phantom erfolgt über die manuelle Positionierung des ersten Markers. Die Abstände zwischen den Markern sind fix, sodass die beiden anderen Marker vom Chirurgen an die gewünschte Stelle rotiert werden können. Anschließend können vom Chirurgen kleinere Anpassungen in der Positionierung vorgenommen werden. Bei dem gesamten Vorgang wird auf die korrekte Einschätzung des Chirurgen vertraut.

Fotouhi et al. \cite{fotouhi2019interactive} haben eine Anwendung entwickelt, mit welcher sich intraoperative Flouroskopien darstellen lassen. Visualisierung von intraoperativen Aufnahmen geschieht häufig über Bildschirme, welche am Fuß eines C-Arm Röntgengerätes angebracht sind, was dazu führt, dass die Hand-Auge-Koordination erschwert wird. Die Anwendung von Fotouhi et al. visualisiert die Aufnahmen in Form eines Pyramidenstumpfes am C-Arm direkt dort, wo die Aufnahme entsteht.  Tracking von C-Arm und Chirurg relativ zum selben Koordinatensystem wird  über zwei Hololenses umgesetzt, wobei eine am C-Arm angebracht ist und eine vom Chirurgen getragen wird. 


\begin{figure}[h!]
	\centering
		\includegraphics[width=0.75\textwidth]{images/pediclescrew.png}
	\caption[Pedikelschrauben Navigation]{Erstellung der intraoperativen Point Cloud; (\textbf{a})Pointer mit Marker; (\textbf{b})Chirurg mit Hololens; (\textbf{c}) Sicht durch Hololens während des Abtastvorgangs; (\textbf{d}) Überlagerung nach der Registrierung \cite{liebmann_pedicle_2019}}
	\label{fig:2.2}
\end{figure}






