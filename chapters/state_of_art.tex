\chapter{\iflanguage{english}{State of the Art}{Stand der Forschung}}
\label{cha:stateOfTheArt}

Im Folgenden werden verwandte Arbeiten genannt, wobei besonderer Fokus auf Anwendungen mit der Microsoft Hololens gelegt wird. Weiterhin wird auf verschiedene Möglichkeiten der Registrierung, u.a. durch Marker Tracking eingegangen. 


\section{Augmented Reality in der Chirurgie}

Augmented Reality kann im medizinischen Bereich in vielerlei Hinsicht eingesetzt werden. Großes Potential haben Anwendungen zur präoperativen Planung \cite{ha_augmented_2016}, intraoperativer Visualisierung und zum Training von Chirurgen \cite{azuma_survey_1997}. Anwendungen mit Head-Mounted-Displays, also Displays, die wie eine Brille auf dem Kopf getragen werden, bieten zusätzlich den Vorteil, dass der Chirurg seine Hände frei nutzen kann. Eine Herausforderung hierbei ist, dass zur Visualisierung die Daten korrekt mit dem Patienten überlagert (\emph{registriert}) werden müssen. Es muss also erreicht werden, dass jeder Punkt auf dem Patienten mit dem korrespondierenden Punkt im angezeigten Objekt übereinstimmt. 

Azuma hat bereits 1997 Anwendungsfälle für AR im medizinischen Bereich genannt, unter anderem ein Projekt, bei welchem eine Biopsie des Brustgewebes durch AR unterstützt wurde. Die Position des Tumors in einem Phantom wurde visualisiert und die Einstichstelle sowie der Pfad der Nadel im Gewebe bis zum Tumor geführt \cite{azuma_survey_1997}.

Birkfellner et al. \cite{birkfellner_head-mounted_2002} nutzen als HMD ein um AR-Fähigkeit erweitertes Varioscope von Life Optics. 

Bei \emph{INPRES} (intraoperative presentation of surgical planning and simulation results)\cite{} wurde als HMD die Optische-Durchsicht-Brille Sony Glasstron verwendet. 

\section{Marker Tracking in der Chirurgie}

Registrierung kann über das Tracken von Markern erreicht werden. Marker sind Objekte, die in der Szene angebracht werden und durch Methoden der Computer Vision im Videostream erkannt werden können. Durch die Position der Marker im Bild lässt sich dann wiederum die Kamerapose ermitteln und so das 3D Objekt an der richtigen Stelle im Raum anzeigen.

Ha und Hong haben ein System zur Unterstützung einer Knochentumorresektion entwickelt. Um den Patienten und die Instrumente zu tracken, wurde ein Tablet Computer mit Kamera genutzt, auf dem auch gleichzeitig die Visualisierung zu sehen war. Als Marker wurden Würfel genutzt, an welchen auf jeder Seite ein planares Bild angebracht war, ähnlich zu Aruco Markern. Diese Marker wurde am Patienten und an den Instrumenten befestigt, ein externes Tracking System war nicht notwendig \cite{ha_augmented_2016}. 


\section{Hololens in der Chirurgie}

Liebmann et al. \cite{liebmann_pedicle_2019} nutzen die Hololens 

Rae et al. \cite{rae_neurosurgical_2018} nutzen die Hololens, um die Platzierung von Bohrlöchern auf dem Schädel anzuzeigen. Mit dem Tool 3D Slicer wird ein 3D Modell erstellt und Marker sowohl an den gewünschten Bohrpunkten als auch an Registrierungspunkten platziert. Die Registrierung des 3D Modells via Hololens mit dem Phantom erfolgt über die manuelle Positionierung der ersten Markers. Die Abstände zwischen den Markern sind fix, sodass die beiden anderen Marker vom Chirurgen an die gewünschte Stelle rotiert werden. Anschließend können vom Chirurgen kleinere Anpassungen in der Positionierung vorgenommen werden. Bei dem gesamten Vorgang wird auf die korrekte Einschätzung des Chirurgen vertraut.







