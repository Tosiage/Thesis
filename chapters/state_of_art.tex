\chapter{\iflanguage{english}{State of the Art}{Stand der Forschung}}
\label{cha:stateOfTheArt}

Im Folgenden werden verwandte Arbeiten genannt, wobei besonderer Fokus auf Anwendungen mit der Microsoft Hololens gelegt wird. Weiterhin wird auf verschiedene Möglichkeiten der Registrierung, u.a. durch Marker Tracking eingegangen. 


\section{Augmented Reality in der Chirurgie}

Augmented Reality kann im medizinischen Bereich in vielerlei Hinsicht eingesetzt werden. Großes Potential haben Anwendungen zur präoperativen Planung \cite{ha_augmented_2016}, intraoperativer Visualisierung und zum Training von Chirurgen \cite{azuma_survey_1997}. Anwendungen mit \emph{Head-Mounted-Displays} (HMD), also Displays, die wie eine Brille auf dem Kopf getragen werden, bieten zusätzlich den Vorteil, dass der Chirurg seine Hände frei nutzen kann. Eine Herausforderung hierbei ist, dass zur Visualisierung die Daten korrekt mit dem Patienten überlagert (\emph{registriert}) werden müssen. Es muss also erreicht werden, dass die wahrgenommene Position in der Szene mit dem computergenerierten Objekt übereinstimmt \cite{birkfellner_head-mounted_2002}. Dazu muss die Kamera kalibriert werden. 

Azuma hat bereits 1997 Anwendungsfälle für AR im medizinischen Bereich genannt, unter anderem ein Projekt, bei welchem eine Biopsie des Brustgewebes durch AR unterstützt wurde. Die Position des Tumors in einem Phantom wurde visualisiert und die Einstichstelle sowie der Pfad der Nadel im Gewebe bis zum Tumor geführt \cite{azuma_survey_1997}.

%Birkfellner et al. \cite{birkfellner_head-mounted_2002} nutzen als HMD ein um AR-Fähigkeit erweitertes Operationsmikroskop, das Varioscope von Life Optics. Dies musste aufwendig kalibriert werden, indem es an ein Gitter montiert wurde und

Bei \emph{INPRES} (intraoperative presentation of surgical planning and simulation results)\cite{} wurde als HMD die Optische-Durchsicht-Brille Sony Glasstron verwendet. 

\section{Marker Tracking in der Chirurgie}

Registrierung kann über das Tracken von Markern erreicht werden. Marker sind Objekte, die in der Szene angebracht werden und durch Methoden der Computer Vision im Videostream erkannt werden können. Durch die Position der Marker im Bild lässt sich dann wiederum die Kamerapose ermitteln und so das 3D Objekt an der richtigen Stelle im Raum anzeigen.

Ha und Hong haben ein System zur Unterstützung einer Knochentumorresektion entwickelt. Um den Patienten und die Instrumente zu tracken, wurde ein Tablet Computer mit Kamera genutzt, auf dem auch gleichzeitig die Visualisierung zu sehen war. Als Marker wurden Würfel genutzt, an welchen auf jeder Seite ein planares Bild angebracht war, ähnlich zu Aruco Markern. Diese Marker wurde am Patienten und an den Instrumenten befestigt, ein externes Tracking System war nicht notwendig \cite{ha_augmented_2016}. 

Kenngott et al. \cite{kenngott_mobile_2018} haben ein mobiles System entwickelt, welches eine möglichst patientennahe Anwendung von Augmented Reality ermöglicht. So können Aufnahmen des Patienten schnellstmöglich visualisiert werden. Dies ist vor allem in Umfeldern, wo es auf Schnelligkeit ankommt (bspw. in der Notaufnahme) ein großer Vorteil gegenüber des üblichen Vorgehens, dem Anzeigen der Aufnahmen an einem dafür vorgesehenen Arbeitsplatz. Um den Bezugsrahmen für die AR Anwendung zu definieren, wurden 15 strahlenundurchlässige Marker genutzt, welche frei auf dem Patienten positioniert werden konnten. Für den Registrierungsprozess wurden 6 dieser Marker genutzt und farblich markiert. Zum Tracking der Marker und Anzeigen des Hologramms wurde ein handelsübliches Apple iPad genutzt.

\cite{frantz_augmenting_2018}

\section{Hololens in der Chirurgie}

Die Microsoft Hololens ist ein HMD für Augmented Reality und zurzeit eines der besten HMDs auf dem Markt geeignet für chirurgische Anwendungen \cite{pratt_through_2018}. Es ist ein optical see-through display, ermöglicht also dem Chirurgen gleichzeitig die Sicht auf die reale Umgebung ohne eine Zeitverzögerung. Die Hololens ist nicht kabelgebunden und lässt sich über Handgesten und Sprachbefehle steuern, was die Sterilität im Operationssaal nicht gefährdet \cite{pratt_through_2018}.

Von Pratt et al. \cite{pratt_through_2018} wurde eine Hololens App zur Unterstützung der Chirurgen bei einer Operation zur Rekonstruktion von Extremitäten entwickelt (siehe Abb. \ref{fig:2.2}). Es wurde festgestellt, dass die Anwendung verlässlicher und weniger zeitaufwändig ist, als die zur Zeit für solche Eingriffe genutzte Doppler-Sonografie. Das Hologramm wurde beim Start der App ausgehend von der Pose der Hololens an einer festgelegten Stelle im Raum gerendert. Die Registrierung wurde vom Chirurgen manuell durch das nutzen der Air-Tap-and-Hold Geste durchgeführt. Über Sprachbefehle oder entsprechende Buttons in einer Toolbar konnte zwischen Translation und Rotation Modus gewechselt werden. 

\begin{figure}[ht]
	\centering
		\includegraphics[width=0.75\textwidth]{images/pediclescrew.png}
	\caption[Pedikelschrauben Navigation]{Erstellung der intraoperativen Point Cloud; (\textbf{a})Pointer mit Marker; (\textbf{b})Chirurg mit Hololens; (\textbf{c}) Sicht durch Hololens während des Abtastvorgangs; (\textbf{d}) Überlagerung nach der Registrierung \cite{liebmann_pedicle_2019}}
	\label{fig:2.1}
\end{figure}

Liebmann et al. \cite{liebmann_pedicle_2019} haben eine Navigationsanwendung für die Platzierung von Pedikelschrauben bei Operationen an der Wirbelsäule entwickelt. Ziel der Anwendung war es, intraoperative Bildgebung wie CT Aufnahmen zu vermeiden, da der Patient dabei einer hohen Strahlenbelastung ausgesetzt wird. Mit 2 der 4 Frontkameras der Hololens auf die über den Research Mode zugegriffen werden kann wurde Marker Tracking zur Bestimmung der Kamerapose umgesetzt. Als Marker wurden sterile kommerziell verfügbare Marker verwendet. Um zu Registrieren, wurde intraoperativ die Oberfläche der Wirbelsäule mit Hilfe eines Pointers, der getrackt werden kann, abgetastet. Zusätzlich wurde der Klicker der Hololens genutzt. Während der Klicker gedrückt wurde, wurde die Position vom Pointer gespeichert. Die so erhaltene intraoperative Point Cloud wurde anschließend vollautomatisch mit der präoperativen Point Cloud registriert (siehe Abb. \ref{fig:2.1}). Abschließend wurde das Hologramm angezeigt und bei korrekter Überlagerung vom Chirurgen bestätigt.  Durch doppelklicken des Hololens Klickers wurde dann die Navigation gestartet. Während der Navigation wurde das Hologramm ausgeblendet und nur die Bohrlöcher visualisiert, zusätzlich wurde dem Chirurgen die aktuelle und gewünschte Neigung eines Navigationswerkzeuges angezeigt.

Rae et al. \cite{rae_neurosurgical_2018} nutzen die Hololens, um die Platzierung von Bohrlöchern auf dem Schädel anzuzeigen. Mit dem Tool 3D Slicer wurde ein 3D Modell erstellt und Marker sowohl an den gewünschten Bohrpunkten als auch an Registrierungspunkten platziert. Die Registrierung des 3D Modells via Hololens mit dem Phantom erfolgte über die manuelle Positionierung des ersten Markers. Die Abstände zwischen den Markern sind fix, sodass die beiden anderen Marker vom Chirurgen an die gewünschte Stelle rotiert werden konnten. Anschließend konnten vom Chirurgen kleinere Anpassungen in der Positionierung vorgenommen werden. Bei dem gesamten Vorgang wurde auf die korrekte Einschätzung des Chirurgen vertraut.

\begin{figure}[ht]
	\centering
		\includegraphics[width=0.75\textwidth]{images/PrattAR.png}
	\caption[Hololensgestützte Extremitätenrekonstruktion]{Hololensgestützte Operation; (\textbf{a}) Blick auf OP über zweite Hololens; (\textbf{b}) Chirurg nutzt Doppler-Sonografie zur Überprüfung; (\textbf{c}) Registrierung von Hologramm und Patient, Hervorhebung wichtiger Positionen durch Pfeile \cite{pratt_through_2018}}
	\label{fig:2.2}
\end{figure}







