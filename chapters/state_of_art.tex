\chapter{\iflanguage{english}{State of the Art}{Stand der Forschung}}
\label{cha:stateOfTheArt}

Im Folgenden werden verwandte Arbeiten genannt, wobei besonderer Fokus auf Anwendungen mit der Microsoft Hololens gelegt wird. Weiterhin wird auf verschiedene Möglichkeiten der Registrierung, u.a. durch Marker Tracking eingegangen. Anschließend werden die verschiedene Marker Sorten und deren Vor- und Nachteile genannt.


\section{Augmented Reality in der Chirurgie}

Birkfellner et al. \cite{birkfellner_head-mounted_2002} nutzen als HMD ein um AR-Fähigkeit erweitertes Varioscope von Life Optics. 

Bei \emph{INPRES} (intraoperative presentation of surgical planning and simulation results)\cite{} wurde als HMD die Optische-Durchsicht-Brille Sony Glasstron verwendet. 



\section{Hololens in der Chirurgie}

Liebmann et al. \cite{liebmann_pedicle_2019} nutzen die Hololens 

Rae et al. \cite{rae_neurosurgical_2018} nutzen die Hololens, um die Platzierung von Bohrlöchern auf dem Schädel anzuzeigen. Mit dem Tool 3D Slicer wird ein 3D Modell erstellt und Marker sowohl an den gewünschten Bohrpunkten als auch an Registrierungspunkten platziert. Die Registrierung des 3D Modells via Hololens mit dem Phantom erfolgt über die manuelle Positionierung der ersten Markers. Die Abstände zwischen den Markern sind fix, sodass die beiden anderen Marker vom Chirurgen an die gewünschte Stelle rotiert werden. Anschließend können vom Chirurgen kleinere Anpassungen in der Positionierung vorgenommen werden. Bei dem gesamten Vorgang wird auf die korrekte Einschätzung des Chirurgen vertraut.

\section{Marker Tracking in der Chirurgie}





