\chapter{\iflanguage{english}{Introduction}{Einführung}}
\label{cha:introduction}

This section should describe your motivation and general introduction to the topic as well as the goals of your work.

\section{Motivation}

Unter \emph{Augmented Reality} wird das Hinzufügen von virtuellen Objekten zur realen Welt verstanden. Diese Objekte können über ein Gerät wie ein Tablet oder ein Head Mounted Display, welches wie eine Brille auf dem Kopf getragen wird, visualisiert werden. Damit die virtuellen Objekte an der korrekten Stelle in der realen Welt angezeigt werden, selbst wenn der Nutzer sich im Raum bewegt, muss die Pose der Kamera bekannt sein. 

-kein hin und her schauen zwischen Patient und Bildschirm
-vermeidung von strahlung

\section{\iflanguage{english}{Goals}{Ziele}}

Getrackt werden soll das Phantom eines Torsos. Es sollen verschiedene Methoden des markerbasierten Trackings implementiert und ein Hologramm der Leber relativ zu den Markern platziert werden. Zu diesen Methoden zählt Vuforia Model Recognition mit einem Trained Model Dataset, bei der die Bauchplatte des Phantoms als Model Target (Marker) fungiert und von Vuforia getrackt wird. Weitere Marker sind Vuforia ImageTargets, ArUco Marker und kreisförmige Farbmarker. Bei diesen drei Methoden werden mehrere Marker auf dem Torso platziert.

