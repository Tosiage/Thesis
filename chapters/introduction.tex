\chapter{\iflanguage{english}{Introduction}{Einführung}}


This section should describe your motivation and general introduction to the topic as well as the goals of your work.

As a placeholder, here are some hints for writing:

\section{Getting Started}
\begin{itemize}
\item First, decide on the language - if you want to write in German, comment out the line '\verb|\selectlanguage{english}|' in \emph{thesis.tex}.
\item We have Texmaker installed on most machines, but you can also use a Tex editor of your choice.
\end{itemize}

\section{Figures and Images}
\begin{itemize}
\item If possible, always use vector graphics.
\item Recommended image formats: PDF, EPS, SVG
\item If you want, you can use the Latex TIKZ package to draw beautiful, scaling images or graphs.
\item Make sure that there is at least one reference to each figure and table somewhere in the text (use \verb|\ref{}|).
\end{itemize}

\section{Texmaker}
You can use any Tex editor and compiler you want to, but here are some hints for Texmaker, which is installed on our system:
\begin{itemize}
\item Open thesis.tex and go to \emph{Options->Define Current Document as 'Master Document'}. This will always compile the full document (even if you're editing a sub-chapter).
\item Go to \emph{Options->Configure Texmaker->Quick Build} to choose what the Quick Build command (F1) should do.
\end{itemize}

\section{General Hints}
\begin{itemize}
\item Cite using the \verb|\cite{}|  Latex command, and put the corresponding bibliography files into \emph{bibliography.bbl}.\\Example: The Dijkstra Algorithm \cite{dijkstra1959note} can be used to find the shortest path between two nodes in a graph.
\item You can emphasize important names in-line by using the \verb|\emph{}| command.
\item Introduce abbreviations when you first use them and consistently use them in the remainder of the text.
\item Try to be done a few days before the deadline so that your supervisor gets a chance to proofread before you hand it in.
\end{itemize}