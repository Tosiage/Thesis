\chapter{Evaluation}
\label{cha:evaluation}

Um die Genauigkeit und Bedienung der Anwendung zu evaluieren, wurde eine Nutzerstudie durchgeführt. Der Ablauf dieser Studie wird im Folgenden erläutert, sowie die Ergebnisse genannt.

\section{Ablauf}

Probanden, welche die Hololens noch nie oder nicht sehr oft genutzt haben, haben zu Beginn das Gesten-Tutorial von Microsoft durchgeführt.
Der Ablauf der Nutzerstudie lässt sich in mehrere Schritte zerlegen. Im ersten Schritt wurde den Probanden die Anwendung mit Hilfe eines bebilderten Textes (siehe \ref{appendix: a}), welcher von der Versuchsleiterin vorgetragen wurde, erläutert. Um eine Eingewöhnung in die Anwendung zu ermöglichen, führten die Probanden die Kalibrierung zunächst an der Vive Pro Box durch. Im dritten Schritt führten die Probanden die Kalibrierung am Phantom durch. Anschließend wurde die mit der Polaris durchgeführte Kalibrierung geladen, welche die Probanden betrachten sollten. Im letzten Schritt wurde von den Probanden ein Fragebogen (siehe \ref{appendix: b}) ausgefüllt, der im ersten Teil Demographische Daten abfragt, sowie wie erfahren die Probanden bereits mit der Hololens waren. Im zweiten Teil wurden Fragen gestellt, die sich mit der Bedienung der Anwendung befassten. Die Kalibrierungen der Probanden wurden zur späteren Auswertung als Datei exportiert.

\section{Ergebnisse}

An der Nutzerstudie haben 13 Probanden teilgenommen. Im Folgenden werden die Ergebnisse des Fragebogens genannt, sowie die Kalibrierungen der Probanden ausgewertet.

\subsection{Fragebogen}

Von den 13 Probanden gaben 38,5\% an, vor dem Tag der Nutzerstudie schon länger als 30 Minuten mit der Hololens gearbeitet zu haben. 84,6\% der Probanden haben das Tutorial zu Gestenerlernung mitgemacht. 92,3\% der Probanden fanden die mit der Polaris durchgeführte Kalibrierung besser als ihre eigene. Mit ihrer eigenen Kalibrierung waren 8 Probanden eher zufrieden und 3 zufrieden und sehr zufrieden (vgl.Abb.\ref{fig:5.1}).  


  \begin{figure}[h!]
    \centering
    \begin{tikzpicture}[scale=1]
      \begin{axis}[
        ybar,
        xtick=data,
        xmin=0.5,
        xmax=6.5,
        x = 2cm,
        ymajorgrids=true,
        bar width=1cm, 
        %xlabel={something else},
        xlabel style={yshift=-1cm},
        xtick align=inside,
        xticklabels={gar nicht zufrieden, nicht zufrieden , eher nicht zufrieden, eher zufrieden, zufrieden, sehr zufrieden},
        %ylabel={something},
        x tick label style={font=\normalsize, rotate=45, anchor=east}
        ]
        \addplot table [x index=0,y index=1, col sep=comma] {fragebogenZufrieden.txt};
      \end{axis}
    \end{tikzpicture}
    \caption[Zufriedenheit mit Kalibrierung]{Wie zufrieden waren sie mit ihrer Kalibrierung?}
    \label{fig:5.1}
  \end{figure}
  
   \begin{figure}[h!]
    \centering
    \begin{tikzpicture}[scale=1]
      \begin{axis}[
        ybar,
        xtick=data,
        xmin=0.5,
        xmax=6.5,
        x = 2cm,
        ymajorgrids=true,
        bar width=1cm, 
        %xlabel={something else},
        xlabel style={yshift=-1cm},
        xtick align=inside,
        xticklabels={gar nicht intuitiv, nicht intuitiv , eher nicht intuitiv, eher intuitiv, intuitiv, sehr intuitiv},
        %ylabel={something},
        x tick label style={font=\normalsize, rotate=45, anchor=east}
        ]
        \addplot table [x index=0,y index=1, col sep=comma] {fragebogenEinfach.txt};
      \end{axis}
    \end{tikzpicture}
    \caption[Einfachheit der Anwendung]{Wie einfach war die Bedienung der Anwendung?}
    \label{fig:5.2}
  \end{figure}

\subsection{Kalibrierungen}

Um die von den Probanden durchgeführten Kalibrierungen auswerten und vergleichen zu können, wurde zunächst pro ImageTarget $t_i$ der Mittelwert $\overline{p}$ aus allen kalibrierten Positionen $p$ und der Mittelwert $\overline{q}$ aus allen kalibrierten Quaternionen $q$ gebildet. Anschließend wurde die euklidische Distanz $d$ zwischen den von den Nutzern kalibrierten Werten und der gemittelten Position berechnet (Abb. \ref{fig:5.3} stellt diese als Boxplot dar). Für jedes $t_i$ wurde die Standardabweichung $\sigma_{pi}$ der Positionen der von den Nutzern kalibrierten Werte berechnet, wobei $c_i$ die Anzahl der Kalibrierungen für das ImageTarget $t_i$ darstellt. \[\sigma_{pi} =\sqrt{ \frac{\sum_{k=1}^{c_i}d^2}{c_i}}\]
Um die kalibrierten Quaternionen vergleichbar zu machen, wurden alle kalibrierten Quaternionen $q$ und $\overline{q}$ in Eulerwinkel $e$ und $\overline{e}$ umgerechnet. Für jede Achse $a$ (wobei $a\in\{x,y,z\}$) wurde der Winkel $\omega$ zwischen $e$ und $\overline{e}$ berechnet (Abbildungen \ref{fig:5.4}, \ref{fig:5.5} und \ref{fig:5.6} stellen diese als Boxplot dar). Für jedes $t_i$ wurde die Standardabweichung $\sigma_{ei}$ der Rotationen um die Achse $a$ bestimmt.
\[\sigma_{e_ai} =\sqrt{ \frac{\sum_{k=1}^{c_i}{\omega_a}^2}{c_i}}\]
Für mit der Polaris durchgeführte Kalibrierung $c_p$ wurde für jedes $t_i$ die euklidische Distanz zu $\overline{p}$ bestimmt, sowie der Winkel $w_a$ zwischen den mit der Polaris erhaltenen Werten für $e_a$ und $\overline{e_a}$. 

%
%
%
%Distanzen zwischen gemittelter Position und kalibrierter Positionen der Nutzer in cm
%
%
\begin{figure}[h!]
\begin{tikzpicture}
\begin{axis}[
boxplot/draw direction=y,
x axis line style={opacity=0},
axis x line*=bottom,
axis y line=left,
enlarge y limits,
ymajorgrids,
y= 1cm,
xtick={1,2,3,4},
xticklabels={ImageTarget 1, ImageTarget 2, ImageTarget 3, ImageTarget4},
x tick label style={font=\normalsize, rotate=45, anchor=east}
]
\addplot+ [
boxplot prepared={
	lower whisker=0.01324115*100,
	lower quartile=0.02097116*100,
	median=0.02396314*100,
	upper quartile=0.04570621*100,
	upper whisker=0.0459336*100,
},
] table [row sep=\\,y index=0]{ 8.654604\\ };
\addplot+ [
boxplot prepared={
	lower whisker=0.01544793*100,
	lower quartile=0.02211648*100,
	median= 0.02404906*100,
	upper quartile=0.04596474*100,
	upper whisker=0.05043853*100,
},
] table [row sep=\\,y index=0]{ 9.534188\\ };
\addplot+ [
boxplot prepared={
	lower whisker=0.01701809*100,
	lower quartile=0.02213652*100,
	median=0.02548886*100,
	upper quartile=0.04344236*100,
	upper whisker=0.04673569*100,
},
] table [row sep=\\,y index=0]{ 9.914161\\ };
\addplot+ [
boxplot prepared={
	lower whisker=0.008535424*100,
	lower quartile=0.017880385*100,
	median=0.025574005*100,
	upper quartile=0.0401248175*100,
	upper whisker=0.0618132*100,
},
] coordinates{  };
\end{axis}
\end{tikzpicture}
\caption[Distanzen]{Distanzen zwischen gemittelter Position und kalibrierter Positionen der Nutzer in cm}
\label{fig:5.3}
\end{figure}

%
%
%
%Winkel zwischen gemittelter Rotation und kalibrierter Rotationen der Nutzer in Grad für X-Achse
%
%
%
\begin{figure}[h!]
\begin{tikzpicture}
\begin{axis}[
boxplot/draw direction=y,
x axis line style={opacity=0},
axis x line*=bottom,
axis y line=left,
enlarge y limits,
ymajorgrids,
y= 0.1cm,
xtick={1,2,3,4},
xticklabels={ImageTarget 1, ImageTarget 2, ImageTarget 3, ImageTarget 4},
x tick label style={font=\normalsize, rotate=45, anchor=east}
]
\addplot+ [
boxplot prepared={
	lower whisker=-7.493694,
	lower quartile=-6.398326,
	median=-5.853821,
	upper quartile=-1.153215,
	upper whisker=-1.153215,
},
] table [row sep=\\,y index=0]{ 31.07181\\ 23.02324\\ };
\addplot+ [
boxplot prepared={
	lower whisker=-6.481321,
	lower quartile= -6.31912,
	median= -5.432692,
	upper quartile=-0.8779526,
	upper whisker=-0.8779526,
},
] table [row sep=\\,y index=0]{ 25.19206\\ 24.75087\\ };
\addplot+ [
boxplot prepared={
	lower whisker=-8.237691,
	lower quartile=-6.896362,
	median=-6.170105,
	upper quartile=-2.451642,
	upper whisker=-2.451642,
},
] table [row sep=\\,y index=0]{33.75811\\ 24.14082\\ };
\addplot+ [
boxplot prepared={
	lower whisker=-10.41908,
	lower quartile=-6.814871,
	median=-5.729482,
	upper quartile= -4.166394,
	upper whisker=-0.9766622,
},
] table [row sep=\\,y index=0]{ 56.47729\\ };
\end{axis}
\end{tikzpicture}
\caption[WinkelX]{Winkel zwischen gemittelter Rotation und kalibrierter Rotationen der Nutzer in Grad für X-Achse}
\label{fig:5.4}
\end{figure}

%
%
%
%Winkel zwischen gemittelter Rotation und kalibrierter Rotationen der Nutzer in Grad für Y-Achse
%
%
%
\begin{figure}[h!]
\begin{tikzpicture}
\begin{axis}[
boxplot/draw direction=y,
x axis line style={opacity=0},
axis x line*=bottom,
axis y line=left,
enlarge y limits,
ymajorgrids,
y= 0.1cm,
xtick={1,2,3,4},
xticklabels={ImageTarget 1, ImageTarget 2, ImageTarget 3, ImageTarget 4},
x tick label style={font=\normalsize, rotate=45, anchor=east}
]
\addplot+ [
boxplot prepared={
	lower whisker=-1.127132,
	lower quartile=-0.716365,
	median=0.3266555,
	upper quartile=1.22898,
	upper whisker=2.720802,
},
] table [row sep=\\,y index=0]{ -9.120387\\ };
\addplot+ [
boxplot prepared={
	lower whisker=-1.432159,
	lower quartile= -0.6351013,
	median= 0.2261093,
	upper quartile=1.245267,
	upper whisker=2.260659,
},
] table [row sep=\\,y index=0]{ -8.767303\\ };
\addplot+ [
boxplot prepared={
	lower whisker=-1.134583,
	lower quartile=-1.001434,
	median=0.04611206,
	upper quartile=0.5393831,
	upper whisker=2.206368,
},
] table [row sep=\\,y index=0]{-13.34515\\ };
\addplot+ [
boxplot prepared={
	lower whisker=-2.20438,
	lower quartile=-1.514614,
	median=-0.858234,
	upper quartile=  0.841911825,
	upper whisker=1.099367,
},
] table [row sep=\\,y index=0]{ 69.01529\\ };
\end{axis}
\end{tikzpicture}
\caption[WinkelY]{Winkel zwischen gemittelter Rotation und kalibrierter Rotationen der Nutzer in Grad für Y-Achse}
\label{fig:5.5}
\end{figure}

%
%
%
%Winkel zwischen gemittelter Rotation und kalibrierter Rotationen der Nutzer in Grad für Z-Achse
%
%
%
\begin{figure}[h!]
\begin{tikzpicture}
\begin{axis}[
boxplot/draw direction=y,
x axis line style={opacity=0},
axis x line*=bottom,
axis y line=left,
enlarge y limits,
ymajorgrids,
y= 0.1cm,
xtick={1,2,3,4},
xticklabels={ImageTarget 1, ImageTarget 2, ImageTarget 3, ImageTarget 4},
x tick label style={font=\normalsize, rotate=45, anchor=east}
]
\addplot+ [
boxplot prepared={
	lower whisker=0.6575012,
	lower quartile=0.8416443,
	median=1.227295,
	upper quartile=2.826284,
	upper whisker=3.747691,
},
] table [row sep=\\,y index=0]{ -26.44675\\ 9.130776\\ };
\addplot+ [
boxplot prepared={
	lower whisker=0.4768982,
	lower quartile=0.6923523 ,
	median=1.134308 ,
	upper quartile=3.146458,
	upper whisker=3.562537,
},
] table [row sep=\\,y index=0]{ -25.24329\\ 7.236584\\ };
\addplot+ [
boxplot prepared={
	lower whisker=-0.07778931,
	lower quartile=1.491058,
	median=1.994385,
	upper quartile=3.450439,
	upper whisker=4.053576,
},
] table [row sep=\\,y index=0]{-33.88715\\ 7.582627\\ };
\addplot+ [
boxplot prepared={
	lower whisker=-4.088603,
	lower quartile=-3.08943525,
	median=-2.669643,
	upper quartile= -0.912190325 ,
	upper whisker=0.2987928,
},
] table [row sep=\\,y index=0]{ 84.44257\\ };
\end{axis}
\end{tikzpicture}
\caption[WinkelZ]{Winkel zwischen gemittelter Rotation und kalibrierter Rotationen der Nutzer in Grad für Z-Achse}
\label{fig:5.6}
\end{figure}


\begin{figure}[h!]
\begin{tabular}{llllll}
\hline
\textbf{ImageTarget $t_i$}  & Samplegröße  & \textbf{$\sigma_{pi}$}  & \textbf{$\sigma_{e_xi}$}  & \textbf{$\sigma_{e_yi}$}  & \textbf{$\sigma_{e_zi}$}  \\
\hline
&\\
\textbf{1} &  11 & 3.924679 & 12.77979 & 2.9669 & 8.603211 \\
&\\
\textbf{2} & 11 & 4.166714 & 11.70274 & 2.835836 & 8.102372\\
&\\
\textbf{3} & 11 & 4.123605 & 13.69464 & 4.153504 & 10.67798 \\ 
&\\
\textbf{4} & 12 & 3.413939 & 17.34399 &  19.95994 & 24.51205 \\
&\\
\textbf{5} & 11 & 4.96984 & 17.22374 &9.408016 & 18.88608 \\
&\\
\textbf{6} & 12 & 4.174401 & 11.9653 & 3.489036 & 8.740802 \\
&\\
\textbf{7} &  11 & 9.924153 & 14.82467& 10.06704 & 32.01934 \\
&\\
\textbf{8} & 11 & 4.298156 & 12.13137 &  3.579294 & 9.036934\\
&\\
\textbf{9}  & 13 & 4.972395 & 2.558084 & 4.701161 & 6.721008 \\
&\\
\textbf{10} & 13 & 5.467957 & 4.51474 & 2.099764 & 9.333352\\
&\\
\textbf{11} & 13 & 5.949328 & 9.868921 & 4.697994 & 10.76875\\
&\\
\textbf{12} & 13 & 4.836598 & 4.185251 & 2.968833 & 6.922363\\
&\\
\textbf{13} & 13 & 8.308828 & 8.117594 & 8.768795 & 11.47547\\
&\\
\textbf{14} & 13 & 6.262945 & 8.958679 & 5.996447 & 16.40697\\
&\\
\textbf{15} & 13 & 6.789291 & 9.637681 & 8.076731 & 13.38344\\
&\\
\textbf{16} & 13 & 5.867448 & 8.075559 & 3.393669 & 7.140728\\
\hline
\end{tabular}
\caption[Standardabweichungen]{Liste aller Standardabweichungen von Durchschnittswert, wobei $\sigma_{pi}$ in cm und $\sigma_{e_xi}$, $\sigma_{e_yi}$, $\sigma_{e_zi}$ in Grad}
\label{fig:5.7}
\end{figure}

