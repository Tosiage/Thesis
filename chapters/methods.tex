\chapter{\iflanguage{english}{Methods}{Methoden}}
\label{cha:methods}

Description of your used methods, algorithms and the actual work you did. This is often the longest and most detailed chapter...

\section{Microsoft Hololens}
-Begriff Hologramm
-Wie findet sich die hololens im raum zurecht (inside out tracking)
-koordinatensysteme
-specs

Das in dieser Arbeit genutzte Optical See-Through HMD ist die Hololens der ersten Generation. Die Hololens wurde 2016 von der Microsoft Corporation auf den Markt gebracht. Sie übertrifft andere kommerziell verfügbaren OST-HMDs in Kontrast und Stabilität der Bildwiederholrate und hat eine geringere Zeitverzögerung \cite{qian_comparison_2017}. Die Hololens ist nicht kabelgebunden, lässt sich über Handgesten sowie Sprachbefehle steuern und gefährdet somit nicht die Sterilität im Operationssaal \cite{pratt_through_2018}.

Ein virtuelles Objekt, welches von der Hololens angezeigt wird, wird \emph{Hologramm} genannt. Die Hololens projiziert ein Hologramm auf zwei Durchsichtdisplays (\emph{Waveguides)}, für jedes Auge eines. Im Gegensatz zu Video See-Through HMDs, welche bei einem Stromausfall dem Nutzer jegliche Sicht versperren, ermöglichen die Waveguides zeitgleich die Sicht auf die reale Umgebung. Für jeden Waveguide gibt es eine \emph{Light Engine}, welche die holographischen Inhalte projiziert.

\textbf{Technische Daten}

\begin{tabular}{llr}
\hline
Komponente    & Daten   \\
\hline
\textbf{Optik}      & zwei Durchsichtdisplays (Waveguides)\\
          & zwei HD 16:9 Light Engines \\
\textbf{Holographische Auflösung} & 2,3 Millionen Lichtpunkte\\
\textbf{Holographische Dichte} & mehr als 2500 Radianten \\
&(2500 Lichtpunkte pro Radiant, wobei $1 rad \approx 57,3^\circ $ ) \\
\textbf{Sensoren}       & stuffed          \\
\textbf{Sprachverstehen}       & stuffed           \\
\textbf{Eingabe/Ausgabe/Konnektivität} & frozen            \\
\textbf{Leistung} & frozen            \\
\textbf{Prozessoren} & frozen            \\
\textbf{Gewicht} & frozen            \\
\textbf{Betriebssystem} & Windows 10           \\
\hline
\end{tabular}



\section{NDI Polaris}


\section{Unity}

\section{Mixed Reality Toolkit}

\section{Vuforia}
-model targets
-model target generator
-image targets
-wie sieht ein gutes image target aus
-die image targets die ich genutzt habe? Oder erst später?

\section{Versuchsaufbau}

Box
Phantom

\section{Kalibrierung}

\section{GUI und Interaktion}
man muss mit dem Hologramm interagieren wegen kalibrierung
zwei möglichkeiten für GUI 
1. box transparent
2. rendering reihenfolge (box muss von achsen überlagert werden)

für 2. entschieden, da bei box transparent buttons drin liegen und so nicht ersichtlich dass interagiert werden kann


