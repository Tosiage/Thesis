\chapter{\iflanguage{english}{Methods}{Methoden}}
\label{cha:methods}

Im Folgenden Kapitel wird eine Übersicht über den Versuchsaufbau und die evaluierten Methoden gegeben. Es wird auf das in dieser Arbeit genutzte SDK (Vuforia) eingegangen. Der Kalibrierungs- und Registrierungsvorgang wird erläutert und zuletzt das User Interface, welches entwickelt wurde.


\section{Aufbau}

\begin{figure}[h!]
	\centering
		\includegraphics[width=1\textwidth]{images/versuchsaufbauBeschriftet.png}
	\caption[Aufbau der Anwendung]{Aufbau der Anwendung; $P$: Phantom; $H$: Hologramm; $i$: Fiducial; $\mathit{poseHMD}$: Pose (Rotation und Position) des HMDs im Raum; $\mathit{poseH}$: ermittelte Pose des Phantoms im Raum und somit Pose, an welcher das Hologramm angezeigt wird; $\mathit{pose_i}$: Pose des Fiducials $i$ im Raum; $\mathit{offset_i}$: Offset des Fiducials $i$ zum Phantom; \textbf{(a)} Model Tracking, Tracking der Bauchplatte anhand von natürlichen Merkmalen; \textbf{(b)} Tracking eines Fiducials}
	\label{fig:4.1}
\end{figure}

Das HMD visualisiert das Hologramm am Phantom. Um das Hologramm an der korrekten Position anzuzeigen, muss die Lage des Phantoms im Raum bekannt sein. Dazu muss das Phantom vom HMD gefunden und verfolgt werden (\emph{Tracking}, vgl.\ref{vuforia}). In dieser Arbeit werden zwei Methoden zum Registrieren von Phantom und Hologramm getestet und evaluiert (vgl. \ref{registrierung}). Die erste Methode ist das Model Tracking (siehe Abb.\ref{fig:4.1}(a)). Dabei wird die Bauchplatte des Phantoms mit Hilfe von Kantendetektion und Abgleich mit einer Datenbank vom HMD gefunden. Das Tracken von natürlichen Merkmalen der Bauchplatte dient zur Stabilisierung. Die Position des Phantoms ist nun bekannt und das Hologramm kann dort angezeigt werden. Model Tracking eignet sich nicht für den Einsatz am Patienten, da für diese Methode ein CAD Model des Objektes zur Verfügung stehen muss, welches viele geometrische Details und natürliche Merkmale aufweist, welche für das Tracking genutzt werden können. Für den Einsatz an einem Phantom bietet es jedoch den Vorteil, dass keine zusätzlichen Fiducials am Phantom angebracht werden müssen.  

Die zweite Methode ist das Tracking des Phantoms mit Hilfe von Fiducials (auch \emph{Marker}). Am Phantom werden Fiducials $i$ angebracht, welche vom HMD gefunden und verfolgt werden können (siehe Abb.\ref{fig:4.1}(b)). Der Offset $\mathit{offset_i}$ zwischen jedem Fiducial und dem Phantom muss bekannt sein. Die Pose des Phantoms und somit auch die Pose des Hologramms $\mathit{poseH}$ wird nun bestimmt, indem die Pose $\mathit{pose_i}$ jedes Fiducials mit dem jeweiligen $\mathit{offset_i}$ multipliziert wird. Sowie Pose als auch Offset beinhalten Position und Rotation. Anschließend wird das Hologramm angezeigt. Um den Offset zwischen Phantom und Fiducial zu bestimmen, muss dieser im Vorfeld kalibriert werden.

Kalibriert wird dieser Offset mittels zwei Methoden, zum einen vom Träger der Hololens selbst, zum anderen mit einem externen Infrarot-Trackingsystem, der Polaris Vega (vgl. \ref{registrierung}). Die Kalibrierung über die Hololens erfolgt indem der Nutzer zu Beginn einen beweglichen, nicht am Phantom befestigten Fiducials, an welchem das Hologramm hängt, auf das Phantom legt. Anschließend kann der Nutzer per Knopfdruck die Position und Rotation des Hologramms verändern. Zuletzt wird für jedes Fiducial $i$ der aktuelle Offset $\mathit{offset_i}$ zum Hologramm abgespeichert. Damit ist die Kalibrierung der Offsets abgeschlossen. 

Alternativ können die Offsets mit der Polaris kalibriert werden. Hierbei werden die Fiducials mit dem Pointer der Polaris abgefahren und so der Offset zu einem am Phantom befestigten passiven Infrarot-Marker bestimmt. Der Offset dieses Infrarot-Markers zum Ursprung des Phantoms ist bekannt, sodass der Offset $\mathit{offset_i}$ der Fiducials zum Phantom berechnet werden kann. Die Kalibrierung mit der Polaris wird unter der Annahme durchgeführt, dass sie genauer ist als die Kalibrierung mit der Hololens und daher zu Evaluationszwecken genutzt werden kann.


\section{Vuforia Augmented Reality SDK} \label{vuforia}

Vuforia \footnote{https://developer.vuforia.com/} ist ein plattformübergreifendes Software Development Kit zur Entwicklung von Mixed Reality Anwendungen. Vuforia lässt sich sowohl mit iOS- und Android-Geräten als auch mit der Hololens nutzen. Es ist in Unity integriert und ermöglicht durch Methoden der Computer Vision das Tracken von Markern in verschiedenen Formen, wie z.B. das Tracken eines einzelnen Bildes, 3D-Konfigurationen mit mehreren Bildern oder eines 3D Objektes. Vuforia ist nicht open source. Die in dieser Arbeit genutzte Version der Vuforia Engine ist 8.0.10.

Vuforia stellt in der Unity Szene ein eigenes Kamera-Asset zur Verfügung, welches in dieser Arbeit Anstelle des MixedRealityCameraParent-Assets aus dem Mixed Reality Toolkit verwendet wird.

\subsection{Vuforia Model Targets}

%https://library.vuforia.com/features/objects/model-targets.html 12.08.2019
\emph{Vuforia Model Targets\footnote{https://library.vuforia.com/features/objects/model-targets.html (Stand 12.08.2019)}} ermöglichen es, Objekte aus der realen Welt basierend auf ihrer Form zu tracken. Dabei kann es sich um ein kleines Objekt wie ein Spielzeugauto handeln, aber auch um ein großes Objekt, wie ein richtiges Auto. Eine Voraussetzung hierfür ist jedoch, dass ein 3D-Modell (bspw. ein CAD-Modell) des Objektes existiert. Damit das Tracking eines Model Targets gut funktioniert, müssen sowohl das Objekt, als auch das CAD-Modell einige Anforderungen erfüllen.   

Ein Objekt, das getrackt werden soll, sollte:
\begin{itemize}
%Fixed Position in Space
\item Unbeweglich im Raum angebracht sein

%Colored or Patterned surface
\item Eine farbige oder gemusterte Oberfläche besitzen

%Sufficient Geometric Detail
\item Eine komplexe geometrische Form mit vielen markanten Punkten besitzen

%non-flexible and rigid
\item Keine beweglichen Teile haben und nicht verformbar sein
\end{itemize}


Das CAD-Modell sollte:
\begin{itemize}
\item Keine Löcher oder Risse aufweisen
\item Keine fehlenden Teile haben
\item Keine Normalen aufweisen, die in eine andere Richtung als die Oberflächennormale zeigen
\item Keine fehlenden Farbinformationen oder fehlende Textur aufweisen
\item Keine falsche Textur aufweisen
\end{itemize}


\textbf{Vuforia Model Target Generator}

Um ein Model Target für die Verwendung in Unity zu erstellen, hat Vuforia den \emph{Model Target Generator} (MTG) entwickelt. Mit Hilfe des MTGs wird aus einem CAD-Modell eine Vuforia Datenbank generiert, welche als \emph{.unitypackage} heruntergeladen und in Unity importiert werden kann. Es ist möglich, die Datenbank mit Cloudbasiertem Deep-Learning zu trainieren und so eine \emph{Advanced Model Target} Datenbank zu erhalten, dies wurde allerdings in dieser Arbeit nicht getan. 

\textbf{Guide Views}

Bei einer untrainierten Datenbank muss das zu trackende Objekt von einer bestimmten Position  und einem bestimmten Winkel aus mit der Hololens angeschaut werden, damit das Tracking beginnen kann. Dazu lassen sich im MTG ein oder mehr Bilder generieren, die während der Anwendung dem Nutzer angezeigt werden. Sie überlagern das Gesehene mit Kanten des 3D-Modells (siehe Abb.\ref{fig:4.1}). Der Nutzer muss sich dann so ausrichten, dass Objekt und Hilfsbild übereinstimmen. So hilft das Bild dem Nutzer so dabei, sich richtig zu positionieren und die Hololens im korrekten Winkel zum zu trackenden Objekt auszurichten. Als \emph{Guide View} wird in Vuforia sowohl das Hilfsbild, als auch Winkel und Position, die relativ zum zu trackenden Objekt eingenommen werden müssen bezeichnet. 

\begin{figure}[ht]
	\centering
		\includegraphics[width=0.75\textwidth]{images/guideView.jpg}
	\caption[Vuforia Guide View]{Guide View (weiße Silhouette) auf einem Android Gerät \footnotemark}
	\label{fig:4.2}
\end{figure}

\footnotetext{https://library.vuforia.com/features/objects/model-targets.html (Stand 13.08.2019)}

\subsection{Vuforia Image Targets}

%https://library.vuforia.com/articles/Training/Image-Target-Guide 14.08.2019
%https://library.vuforia.com/content/vuforia-library/en/articles/Training/Extended-Tracking.html
Ein \emph{Vuforia Image Target\footnote{https://library.vuforia.com/articles/Training/Image-Target-Guide (Stand 14.08.2019)}} ist eine Art Fiducial, das von Vuforia getrackt werden kann. Dabei muss zuerst eine Datenbank im Target Manager des Vuforia Developer Portals angelegt werden. Anschließend kann diese als \emph{.unitypackage} heruntergeladen und in das Unityprojekt importiert werden. Die Vuforia Engine findet und trackt die natürlichen Merkmale in einem Target indem sie diese mit der Datenbank abgleicht. 

Vuforia trackt ein Target solange es teilweise im Sichtfeld der Kamera ist. Ist das Target nicht mehr im Sichtfeld der Kamera und kann somit nicht mehr getrackt werden. Mit Hilfe von \emph{Extended Tracking\footnote{https://library.vuforia.com/content/vuforia-library/en/articles/Training/Extended-Tracking.html (Stand 14.08.2019)}} kann die Robustheit des Trackings erhöht werden. Extended Tracking bedeutet, dass die Pose des Targets bekannt bleibt, obwohl das Target sich nicht mehr im Sichtfeld des Gerätes befindet. Dies wird mit Hilfe des \emph{Positional Device Trackers} erreicht, der seit der Vuforia Engine 7.2 standardmäßig aktiviert ist. Durch Extended Tracking ist es möglich, komplexe Hologramme anzuzeigen, bei welchen das Target durch die Hologrammgröße bei Betrachtung nicht konstant im Sichtfeld ist oder bei welchen das Target durch Nutzung von Handgesten stellenweise verdeckt wird.

\textbf{Image Target Optimierung}

Das Vuforia Developer Portal bietet eine Bewertung von 0 bis 5 Sternen für hochgeladene Bilder an. Diese Bewertung sagt aus, wie gut sich das Bild als Image Target eignet, also wie gut das Bild mit Hilfe von Vuforia erkannt und getrackt werden kann. Je höher die Bewertung, desto besser eignet es sich. Ein Bild, welches als Image Target verwendet werden soll, sollte:
\begin{itemize}
\item detailliert sein,
\item hohen Kontrast aufweisen und
\item keine sich wiederholenden Muster haben.
\end{itemize} 

Vuforia analysiert ein hochgeladenes Bild auf seine Merkmale und speichert Bild und Merkmale in einer Datenbank ab. Unter Merkmal versteht Vuforia ein 'scharfes, spitzes, kantiges Detail im Bild'. Desweiteren bezeichnet Vuforia die Merkmale als 'kontrastbasierte Merkmale'. Da Vuforia nicht open source ist, kann keine genauere Aussage über den von Vuforia genutzten Computer Vision Algorithmus zur Merkmalserkennung getroffen werden, es lässt sich jedoch vermuten, dass es ein Algorithmus ähnlich zu SIFT, SURF oder ORB ist.
Die gefundenen Merkmale eines Bildes werden im Target Manager des Vuforia Developer Portals als gelbe Kreuze angezeigt. 

\newpage

\textbf{Genutzte Image Targets}

Die in dieser Arbeit genutzten Marker wurden mit einem AR Marker Generator \footnote{https://shawnlehner.github.io/ARMaker/ (Stand 06.07.2019)} erstellt. Ihnen wurde eine Nummer in der linken oberen Ecke hinzugefügt, um sie unterscheidbar zu machen. Sie wurden auf nicht-glänzendem Papier in der Größe 5cm x 5cm gedruckt. Damit sie nicht flexibel sind, wurden sie auf einem Stück Pappe befestigt. Im Target Manager des Vuforia Developer Portals haben sie eine 5-Sterne-Bewertung erhalten, eignen sich also sehr gut als Image Targets.


\begin{figure}[ht]
	\centering
		\includegraphics[width=1\textwidth]{images/features.png}
	\caption[Marker Features]{Links: Merkmale im Beispielbild von Vuforia, Rechts: Merkmale der in dieser Arbeit verwendeten Marker}
	\label{fig:4.3}
\end{figure}

\section{Registrierung} \label{registrierung}

Ziel dieser Arbeit ist, ein Hologramm mit Hilfe der Hololens ohne externes Tracking System an einem Phantom zu Registrieren. Dazu muss das Phantom während der Laufzeit von der Hololens getrackt werden können. Implementiert wird Vuforia Model Target Tracking und Vuforia Image Target Tracking, worauf im Folgenden genauer einegegangen wird.

%TODO wohin damit? related work vll am ehesten, wie cylinder targets
%Eine weitere Überlegung galt kreisförmigen einfarbigen Point Fiducials. In diesem Fall hätte auf den Kamerastream der Hololens zugegriffen werden müssen. In diesem hätten erst alle Pixel in der Farbe der Fiducials gefunden werden müssen und anschließend beispielsweise mit Hilfe der Hough-Transformation die Fiducials segmentiert werden müssen. Ein Vorteil bei kreisförmigen Fiducials ist, dass die Kreisform relativ invariant zu Verzerrungen ist. Allerdings käme es durch Unterschiede im Lichteinfall zu Farbvarianzen im Kamerastream, sodass es Probleme beim finden der korrekten Farbpixel gegeben hätte. Zusätzlich bietet ein kreisförmiger Fiducial keine 6DoF, sodass zu jeder Zeit mehrere Fiducials sichtbar sein müssten, um die Rotation der Kamera korrekt bestimmen zu können. Aus diesem Grund wurde auch diese Möglichkeit verworfen.



\subsection{Registrierung über Model Target Tracking am Phantom}


Es wird ein Vuforia Model Target Tracking implementiert. Getrackt wird die Bauchplatte des Phantoms, da nur diese vollständig unbedeckt sichtbar ist. Das CAD-Modell der Bauchplatte wird in den Model Target Generator geladen. Im Model Target Generator werden Maße des Modells angegeben. Damit das Tracking beginnen kann, muss das Modell mit der Hololens aus einem bestimmten Winkel und von einer bestimmten Position aus angeschaut werden. Diese Posen werden im Modell Target Generator festgelegt, sowie die dazugehörigen Guide Views erstellt, um den Träger der Hololens beim Einnehmen dieser Positionen zu unterstützen. Es werden insgesamt vier Posen festgelegt, aus denen das Modell erkannt werden kann: von der linken Seite aus, von der rechten Seite aus, sowie vom oberen und unteren Ende des Phantoms (siehe Abb.\ref{fig:4.4}). Während der Laufzeit der Anwendung erkennt Vuforia, wenn das Phantom im Sichtfeld der Hololens ist und zeigt den dieser Perspektive entsprechenden Guide View an. Bewegt sich der Nutzer um das Phantom, wird automatisch zum der neuen Position entsprechenden Guide View gewechselt. Positioniert der Nutzer sich so, dass Guide View und Phantom übereinstimmen, beginnt das Tracking des Phantoms und das Hologramm wird angezeigt.


\begin{figure}[ht]
	\centering
		\includegraphics[width=0.75\textwidth]{images/guideviews.png}
	\caption[Guide Views]{Guide Views der Anwendung; Oben: Guide Views der Seiten; Unten: Guide Views vom unteren und oberen Ende}
	\label{fig:4.4}
\end{figure}


Das Model Tracking funktioniert gut, solange der Nutzer in einer Entfernung vom Phantom steht, in der die Hololens die Bauchplatte vollständig im Sichtfeld hat. Geht der Nutzer näher an das Phantom heran, stoppt das Tracking und das Hologramm wird nicht mehr an der korrekten Position angezeigt. Da das Ziel jedoch das ist, das Hologramm auch aus der Nähe betrachten zu können, sind Fiducials besser geeignet.



\subsection{Registrierung über Fiducials} 

Die Registrierung von Hologramm und Objekt in der realen Welt wird mit Hilfe von Vuforia Image Targets als Fiducials durchgeführt. Für Testzwecke wurde zunächst ein einfacher Quader als zusätzliches Objekt genutzt, an welchem ein in Unity aus einem Würfel erstelltes 3D-Modell registriert wurde.  

An Box und Phantom werden Fiducials angebracht (Vgl. Abb.\ref{fig:4.3}). Sie werden so befestigt, dass sie mit der Hololens von möglichst vielen Positionen aus gut sichtbar sind. 
Beim Phantom werden die Fiducials nicht auf der Bauchplatte angebracht, da diese abnehmbar ist und somit eine vorher durchgeführte Kalibrierung durch eine verschoben abgelegte Bauchplatte ungenau werden würde.

Jedes Fiducial $i$ besitzt nach seiner Kalibrierung einen Positions- und Rotationsoffset in Form einer Transformationsmatrix $\mathit{offset_i}$ vom Fiducial zum Ursprung des entsprechenden Modells (Box oder Phantom). Jedes Fiducial kennt also die Pose, an der das 3D-Modell in Relation zu sich selbst liegt. Wird ein Fiducial während der Laufzeit erkannt und getrackt, wird dieser Offset zur Bestimmung der Transformationsmatrix $\mathit{poseH}$ des Hologramms genutzt.

 \[poseH_i = \mathit{offset_i} \cdot pose_i\] 

Da durch die planare und quadratische Form eines Fiducials seine Orientierung im Raum eindeutig festgestellt werden kann, wäre ein einzelnes Fiducial theoretisch ausreichend, um das Hologramm korrekt zu positionieren. Da sich der Träger der Hololens jedoch um das Phantom bzw. die Box herum bewegen kann, ist ein Fiducial nicht durchgehend sichtbar. Es kommt zusätzlich vor, dass aufgrund eines zu spitzen Winkels zwischen Fiducial und Hololenskamera die Rotation des Fiducials von Vuforia nicht korrekt erkannt wird. Aus diesen Gründen werden mehrere Fiducials angebracht. Es wird aus den Offsetvektoren aller zurzeit getrackten und kalibrierten Fiducials, sowie auch aus den Offsetquaternionen dieser Fiducials das arithmetische Mittel gebildet. Ziel davon ist es, die Positionierung stabiler und genauer gegenüber eventuellen Abweichungen im Tracking oder falscher Kalibrierung eines Fiducials zu machen.

\begin{figure}[ht]
	\centering
		\includegraphics[width=1\textwidth]{images/BoxPhantomMarker.png}
	\caption[Fiducials an Box und Phantom]{Links: Marker an der Box, Rechts: Fiducials am Phantom}
	\label{fig:4.5}
\end{figure}

\textbf{Mittelwert von Quaternionen}

Um den Mittelwert von mehr als zwei Quaternionen zu bestimmen, muss getestet werden, ob der Quaternion im Vergleich zu den anderen Quaternionen invertierte Vorzeichen hat, da $q$ und $-q$ die gleiche Rotation beschreiben, aber nicht gemittelt werden können. Um dies herauszufinden, kann das Skalarprodukt der beiden Quaternionen verwendet werden. Das Skalarprodukt gibt den Cosinus des Winkels zwischen den beiden Quaternionen an. Ist der Cosinus kleiner als $0$, müssen die Vorzeichen des Quaternions invertiert werden. Anschließend werden die Quaternionen komponentenweise addiert und gemittelt. Zum Schluss wird der gemittelte Quaternion normalisiert 
\footnote{http://wiki.unity3d.com/index.php/Averaging\_Quaternions\_and\_Vectors (Stand 02.09.2019)}
\footnote{https://docs.unity3d.com/Manual/UnderstandingVectorArithmetic.html (Stand 02.09.2019)}.

\begin{figure}[h!]
	\centering
		\includegraphics[width=0.75\textwidth]{images/poseberechnung.png}
	\caption[Registrierung]{Beispiel der Berechnung der Hologramm-Pose $poseH$ mit zwei sichtbaren Fiducials $1$ und $2$. Die Position $pose_i$ und der Offset $\mathit{offset_i}$ beider Fiducials sind bekannt. Somit kann für jedes Fiducial eine Pose für das Phantom $poseH_i$ geschätzt werden. Durch die Mittelung wird die Position $poseH$ berechnet an der das Phantom angezeigt wird.}
	\label{fig:4.6}
\end{figure}

\section{Kalibrierung} \label{kalibrierung}

Damit das 3D-Modell, welches angezeigt werden soll, positioniert werden kann, müssen die Marker kalibriert werden, also Offsetrotation und Offsetposition der Fiducials bekannt werden. Diese Kalibrierung kann auf zwei Wege vorgenommen werden: mit Hilfe der Hololens direkt in der Anwendung oder mit Hilfe eines externen Trackingsystems wie der Polaris. 

\subsection{Kalibrierung mit der Hololens}

Zu Beginn der Kalibrierung mit der Hololens hängt das Hologramm an einem Fiducial, das nicht am Phantom angebracht ist, so dass der Nutzer das Phantom beliebig im Raum bewegen kann. Mit Hilfe dieses Fiducialss kann das Hologramm frei im Raum bewegt werden. Der Träger der Hololens nutzt diesen Marker nun dazu, das Hologramm so gut wie möglich auf dem Objekt (Box oder Phantom) zu positionieren. Der Marker wird auf dem Objekt abgelegt. Sobald einer der fest am Objekt angebrachten Marker im Sichtfeld der Hololens ist, kann die tatsächliche Kalibrierung per Knopfdruck gestartet werden. 

Das Hologramm hängt nun nicht mehr an dem beweglichen Marker, sondern an den Markern die zum Zeitpunkt des Knopfdrucks sichtbar waren. Nun ist es möglich, das Hologramm über ein Interface entlang X-, Y- oder Z-Achse zu verschieben oder um eine Achse zu rotieren. Ist der Nutzer zufrieden mit der Ausrichtung des Hologramms, wird ein \emph{Save-Button} betätigt, welcher den Offset in Rotation und Position der zu diesem Zeitpunkt von der Kamera sichtbaren Marker pro Marker abspeichert. Diese Marker gelten nun als kalibriert und werden von nun an zur Positionierung des Hologramms genutzt.

Der Nutzer kann nun von einer anderen Position im Raum schauen, ob die Position des Hologramms korrekt ist. Ist sie dies nicht, kann weiterhin über das Interface eine Feinjustierung der Position und Rotation des Hologramms vorgenommen werden. Anschließend muss der Save-Button betätigt werden, um wiederum für alle zu diesem Zeitpunkt sichtbaren Marker den Offset zu speichern und sie als kalibriert zu markieren. Ist ein bereits als kalibriert markierter Marker zu dem Zeitpunkt sichtbar, so wird der Offset überschrieben. Dieser Vorgang wird solange wiederholt, bis alle Marker als kalibriert markiert sind. Anschließend kann die Kalibrierung gespeichert und exportiert werden. Bei Neustart der Anwendung kann ebenso eine schon vorher abgeschlossene Kalibrierung geladen werden.

\textbf{Anpassung der Position über das Interface}

Zur Anpassung der Pose (Position und Rotation) über das Interface wurde ein temporärer Offset (\emph{tempOffset}) eingeführt, welcher durch die Buttons des Interfaces geändert werden kann. Der Algorithmus zur Anpassung der Pose wurde in zwei Varianten implementiert. Diese zwei Varianten sehen in Pseudo-Code wie folgt aus: 

\textbf{Variante 1}

\begin{itemize}
\item[1] Initiale Positionierung des Hologramms mit Hilfe des beweglichen Markers
\item[2] Sobald mindestens ein anderer Marker sichtbar ist, Start des Kalibrierungsprozesses über Button \emph{Start Calibration}, nutzen der zu diesem Zeitpunkt sichtbar gewesenen Marker zur Positionierung des Hologramms

\newpage

\item[3] Für als kalibriert markierte Marker gilt 
 \[poseH_i = \mathit{tempOffset_i} \cdot \mathit{offset_i} \cdot pose_i\] 
 wobei \emph{markerOffset} der bereits kalibrierte Offset ist
\item[4] Anpassen von \emph{tempOffset} über Interface Buttons
\item[5] Speichern: für alle in diesem Moment sichtbaren Marker gilt 
\[\mathit{offset_i} = \mathit{tempOffset_i} \cdot \mathit{offset_i}\] 
Anschließend setzen von $\mathit{tempOffset_i}$ auf $identity$ für alle Marker
\end{itemize}

Wie in Abb. \ref{fig:4.7} zu sehen ist, kann es durch diese Methode zu großen Sprüngen des Hologramms kommen, wenn Marker im Sichtfeld sind, von denen einige einen $\mathit{tempOffset}$ besitzen, bei anderen jedoch der $\mathit{tempOffset} = identity $ und somit äquivalent zu einem nicht vorhandenen Offset ist. Aus diesem Grund wurde Variante 2 implementiert. Statt für jedes Fiducial $i$ einen eigenen $\mathit{tempOffset_i}$ zu speichern, wird der $\mathit{tempOffset}$ auf die bereits gemittelte $poseH$ der zur Positionierung genutzten Fiducials aufgerechnet.

\textbf{Variante 2}

\begin{itemize}
\item[1] Initiale Positionierung des Hologramms mit Hilfe des beweglichen Markers
\item[2] Sobald mindestens ein anderer Marker sichtbar ist, Start des Kalibrierungsprozesses über Button \emph{Start Calibration}, nutzen der zu diesem Zeitpunkt sichtbar gewesenen Marker zur Positionierung des Hologramms
\item[3] Für als kalibriert markierte Marker gilt 
 \[poseEstimationH_i = \mathit{offset_i} \cdot pose_i\] 
 \[poseH = \mathit{tempOffset} \cdot poseEstimationH\]
\item[4] Speichern: für alle in diesem Moment sichtbaren Marker wird der aktuelle Offset von Hologramm zu Marker abgespeichert und $\mathit{tempOffset} = identity$ gesetzt
\end{itemize} 

Zu Sprüngen wie in Fall 1 kommt es mit dieser Methode zwar immer noch, jedoch fallen diese deutlich kleiner aus (vgl. Abb. \ref{fig:4.8}) und verbessern das Nutzererlebnis so erheblich.

\begin{figure}[h!]
	\centering
		\includegraphics[width=0.75\textwidth]{images/Kalibrierung1.png}
	\caption[Kalibrierung Methode 1]{Berechnung der Hologrammpose $poseH$ während des Kalibrierungsvorgangs mit zwei sichtbaren Fiducials. Der temporäre Offset $\mathit{tempOffset_i}$ wird auf den $\mathit{offset_i}$ aufmultipliziert. Durch Mittelung der geschätzten Posen $poseH_i$ wird die Hologrammpose $poseH$ bestimmt.}
	\label{fig:4.7}
\end{figure}


\begin{figure}[h!]
	\centering
		\includegraphics[width=0.75\textwidth]{images/Kalibrierung2.png}
	\caption[Kalibrierung Methode 2]{Berechnung der Hologrammpose $poseH$ während des Kalibrierungsvorgangs mit zwei sichtbaren Fiducials. Der temporäre Offset $\mathit{tempOffset}$ wird auf die gemittelte Pose $poseEstimationH$  aufmultipliziert, wodurch die Hologrammpose $poseH$ erhalten wird.}
	\label{fig:4.8}
\end{figure}

\subsection{Kalibrierung mit der Polaris}

Es wurde zusätzlich eine Kalibrierung mit Hilfe der Polaris ausgeführt. Die Polaris besitzt ein kalibriertes Pointer-Tool, welches von der Polaris getrackt werden kann und bei dem der Abstand von Infrarot-Marker zur Spitze des Tools bekannt ist. Per Knopfdruck kann die aktuelle Position der Pointer-Spitze in Polaris-Koordinaten abgespeichert werden. Am Phantom ist ebenso ein Polaris-Marker angebracht.

Ziel ist es, den Offset $o_{iP}$ von jedem Fiducial $i$ zum Ursprung $P$ des Phantoms zu bestimmen. Dazu muss im ersten Schritt festgestellt werden, wie der Polaris-Marker $M$ relativ zum Ursprung des Phantoms liegt. Schon bekannt ist, wie gewisse Punkte auf dem Phantom relativ zu seinem Ursprung liegen. Die Punkte wurden mit dem Pointer in einer bestimmten Reihenfolge abgefahren und ihre Positionen in Polaris-Koordinaten abgespeichert. So wird bekannt, wie diese Punkte relativ zum Polaris-Marker liegen. Anschließend kann daraus berechnet werden, wie der Polaris-Marker relativ zum Ursprung des Phantoms liegt (Transformation $o_{MP}$).

Im zweiten Schritt wurde auf die gleiche Weise die Transformation $o_{Mi}$ für jedes einzelne Fiducial $i$ zum Polaris Marker berechnet. Mit dem Pointer wurden die vier Ecken der Vuforia Marker abgefahren, wodurch anschließend bestimmt werden konnte, wie der Ursprung jedes Vuforia Markers zum Polaris-Marker liegt.  

Nun ist die Transformation $o_{MP}$ vom Polaris-Marker $M$ zum Phantom-Ursprung $P$ bekannt, sowie die Transformation $o_{Mi}$ vom Polaris-Marker $M$ zu den einzelnen Fiducials $i$. Um nun die Transformation $o_{iP}$ der Fiducials $i$ zum Phantom-Ursprung $P$ zu bestimmen, muss die Transformation von Polaris-Marker zu Fiducial invertiert $o_{Mi}^{-1}$) und mit der Transformation $o_{MP}$ von Polaris-Marker zu Phantom-Ursprung multipliziert werden. Im letzten Schritt werden die so erhaltenen Transformationen aus dem Polaris-Koordinatensystem in das Unity-Koordinatensystem übertragen, da das Polaris-Koordinatenssystem linksdrehend mit der negativen X-Achse von der Polaris aus nach oben zeigend ist und in Millimetern rechnet, während das Unity-Koordinatensystem rechtsdrehend mit der Y-Achse in Welt-Koordinaten nach oben zeigend ist und in Metern rechnet.


\begin{figure}[h!]
	\centering
		\includegraphics[width=0.75\textwidth]{images/KalibrierungPolaris.png}
	\caption[Kalibrierung Polaris]{Kalibrierung mit der Polaris: Transformation $o_{MP}$ von Polaris Marker $M$ zum Ursprung $P$ des Phantoms, Tranformation $o_{Mi}$ vom Polaris Marker $M$ zum Fiducial $i$, sowie die gesuchte Transformation $o_{iP}$ vom Fiducial $i$ zum Ursprung $P$ des Phantoms.}
	\label{fig:4.9}
\end{figure}

\section{GUI und Interaktion}

Um bei der Kalibrierung mit der Hololens das Hologramm genauer ausrichten zu können, wurde ein Interface entwickelt. Dieses Interface besteht aus den visualisierten Achsen des lokalen Koordinatensystems des jeweiligen Hologramms (Box oder Phantom). Über drei Buttons kann eine Achse ausgewählt werden. Ist eine Achse ausgewählt, werden die anderen beiden Achsen aus- und Buttons in Form von Pfeilen eingeblendet (siehe Anhang \ref{appendix: a}). Mit Hilfe dieser Buttons ist es möglich, das Hologramm entlang der ausgewählten Achse in 1-cm-Schritten zu verschieben und um diese Achse in 1-Grad-Schritten zu rotieren. 

Weiterhin gibt es zwei Möglichkeiten, das Interface dem Nutzer anzuzeigen:

\begin{itemize}
\item Das Material des Hologramms erhält Transparenz und das Interface wird in das Hologramm hineingelegt oder
\item Das Hologramm wird ähnlich wie Ebenen in einem Bildbearbeitungsprogramm auf eine Ebene gelegt, welche von allen anderen Ebenen überlagert wird. Realisieren lässt sich dies in Unity, indem eine zweite Kamera der Szene hinzugefügt wird, welche nur das Hologramm (Phantom oder Box) rendert. Die ursprüngliche Kamera rendert alles in der Szene außer dem Hologramm.
\end{itemize}

Ein Nachteil der ersten Variante ist, dass aufgrund des transparenten Materials der Box der Eindruck entstehen könnte, dass mit den angezeigten Pfeilen an den Achsen nicht interagiert werden kann, da sie innerhalb des Hologramms liegen. Da dieser Eindruck bei Variante zwei nicht entsteht, da das Koordinatensystem auf dem Hologramm gerendert wird, wurde diese Variante weiterhin genutzt.

